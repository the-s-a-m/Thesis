\chapter{Method}

\section{Concept}

\subsection{K-Means}
Points are assigned to different clusters by using the minimal distances. The following approach is used:

\begin{enumerate}
    \item First, random centroids are created.
    \item Then every point is assigned to the nearest centroid by calculating the euclidean distance.
    \item Then centroids are moved to the center of their assigned graph points.
    \item Till no point is moved or the maximum iterations is reached the process is repeated from item 2.
\end{enumerate}

\subsubsection{Connected Cluster Problem} \label{sec:kmenasProblem}
The K-Mean algorithm is based on the euclidean distance between cluster centroids and street junctions, the edge data (e.g. street length) is not used. As a result it leads to unexpected transitions between clusters.

\subsubsection{Connected Cluster Approach} \label{sec:connected_cluster_approach}
To solve the problem of unexpected transitions between clusters the best result of the K-Means algorithm can be combined with the distance measurement of a \acrlong{APSP} (\acrshort{APSP}) algorithm (Dijkstra / Floyd-Warshall). These algorithms are used for the hierarchical clustering algorithms \ref{sec:shortest_path}. 

