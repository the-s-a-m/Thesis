\chapter{Introduction}
In this thesis the initial task was to learn how street network and buildings can be described as grammar for later analysis and regeneration. There exist many different approaches like Shape Grammar \ref{sec:shape_grammar} (building faces generation), L-Systems \ref{sec:L-Systems} (plants growing), Space Syntax \ref{sec:space_syntax} and many others. 

The \textit{Chair of Information Architecture} provided a street generation and analysing tool named CPlan \ref{CPlan}. To become acquainted with the application and to allow using the existing genetic algorithms a tree generating algorithm was developed by us. 

Selecting interesting areas from different cities and recombining them into a new one would allow a fast creation of new cities. To reach this aim many steps are needed. First of all the cities must be separated into reasonable parts based on vertex position or edge length. The approach of this thesis is to use clustering algorithms from the area of machine learning. Then the different area should be valued to select useful areas.

In this document flat \ref{K-Means} (K-Means) and hierarchical \ref{hierarchicalClustering} (WPGMA, UPGMA) clustering algorithms where compared \ref{sec:measurements-speed} and extended to correct wrong assignments \ref{K-Means_shortest_path}. %TODO memory footprint here!

The implemented clusters are measurements \ref{sec:measurements-cluster-analysis} based on the suggestions provided by the ETH-Zurich. For further use the results can now be exported into a JSON-File. %TODO more here

Student at the ETH is working on an regrow algorithm to recombine the clusters to reasonable city.
%TODO more here