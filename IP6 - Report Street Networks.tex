\documentclass[11pt, a4paper]{report}

\usepackage{graphicx}
\graphicspath{ {images/} }
\usepackage{caption}
\usepackage{subcaption}

\usepackage[utf8]{inputenc}
\usepackage{csquotes}

\usepackage[square,sort,comma,numbers]{natbib}
\bibliographystyle{IEEEtran}

\usepackage{amsmath}
\newcommand\addtag{\refstepcounter{equation}\tag{\theequation}}
\usepackage{txfonts}

\usepackage{multirow}

\setlength{\parindent}{0ex}
\setlength{\parskip}{1ex}

%Rechtangle painting
\usepackage{tikz}
\usepackage[framemethod=tikz]{mdframed}

\usepackage{hyperref}
\usepackage{glossaries}
\makenoidxglossaries

\mdfdefinestyle{mdthight}{
    linewidth=1pt,
    innerleftmargin=0bp,
    innerrightmargin=0bp,
    innertopmargin=0bp,
    innerbottommargin=0bp
}

\loadglsentries{Glossary}

\begin{document}

\title{
    \includegraphics[width=1.75in]{fhnw_fhnw_logo_en.png} \\
    \vspace*{1in}
    \textbf{Report Street Networks}}
\author{
    Authors: \\
    Samuel Merki\\
    Janis Peyer\\
    \vspace*{0.4in} \\
    Coordinator: \\
    Prof. Dr. Stefan Arisona
    \vspace*{0.4in} \\
    Client: \\
    Chair of Information Architecture,
    ETH Zürich
    \vspace*{0.4in} \\
    Course of Studies: \\ Bachelor of Science in Computer Science
    \vspace*{0.4in} \\
    Institute of 4D Technologies \\
    \textbf{University of Applied Sciences and Arts  }\\
    Northwestern Switzerland FHNW
} \date{Friday 19, 2016}
\maketitle
\setcounter{page}{1}

\tableofcontents

\input{050_Abstract}

\chapter{Introduction}
In this thesis the initial task was to learn how street networks and buildings can be described as grammar for later analysis and regeneration. There exist many different approaches like Shape Grammars \ref{sec:shape_grammar} (building faces generation), L-Systems \ref{sec:L-Systems} (plants growing), Space Syntax \ref{sec:space_syntax} and many others. 
Because the application CPlan \ref{CPlan} did not provide the ability to use grammars and our customer preferred a graph based approach all later described algorithms are based on an unidirectional graph.

The \gls{iA} provided a street generation and analysing tool named CPlan \ref{CPlan}. To become acquainted with the application and to allow using the existing genetic algorithms a tree generating algorithm was developed.

Selecting interesting areas from different cities and recombining them into a new one would allow a fast creation of new cities. To reach this aim many steps are necessary. First of all the cities must be separated into reasonable parts based on vertex position or edge length. The approach of this thesis is to use clustering algorithms from the field of machine learning. Useful cluster areas can then be selected by comparing different measured values like the variance of the street length or the median block length.

For this thesis the centroid-based \ref{sec:K-Means} clustering algorithm K-Means was implemented and tested. To correct wrong assignment a shortest path algorithm was used \ref{sec:K-Means_shortest_path} to create connected cluster subgraphs.

Another approach are the hierarchical clustering algorithms \ref{sec:hierarchicalClustering} were different reduction formula exist. First the Single-Linkage formula was realised. Unfortunately only one cluster in the center with many small clusters at the surrounding was created.

%Quickfix
\newpage

Then the reduction formula \acrshort{WPGMA} (\acrlong{WPGMA}) was implemented and tested \ref{sec:UPGMAandWPGMA}. The city areas where separated but they vary widely in the size. To correct this issue the \acrshort{UPGMA} (\acrlong{UPGMA}) reduction formula was used \ref{sec:UPGMAandWPGMA} because every edge counts equal. This approach created better clusters but the cluster sizes were still too different.

Equal clusters sizes then where achieved by modifying the output \ref{sec:outout_modification}. This means instead of splitting the hierarchy as it was created, always the cluster with the most notes was split. As a result the areas where equal sized as preferred in city clustering. The result can be viewed in Figure  \ref{fig:cluster_with_mod_sizes}.

\begin{figure}[ht]
    \centering
    \begin{mdframed}[style=mdthight, userdefinedwidth=0.4\linewidth, align=center]
        \includegraphics[width=\linewidth]{modified_cluster_size.png}
    \end{mdframed}
    \caption{Clusters with modified size}
    \label{fig:cluster_with_mod_sizes}
\end{figure}

During tests with big cities the hierarchical clustering methods had a hight memory usage. To reduce the memory footprint \ref{sec:memory_usage} optimisations were made and specialized data structures were used.

The created clusters were then measured \ref{sec:measurements-cluster-analysis} and compared based on the suggestions \ref{sec:clusterRating} provided by the ETH-Zurich. Additionally the results can now be exported into a JSON-File for further use.

To recombine the separated areas/districts to a new city another project at ETH-Zurich is currently working on a regrow algorithm \ref{sec:future_work}. 

\chapter{Introduction}
In this thesis the initial task was to learn how street network and buildings can be described as grammar for later analysis and regeneration. There exist many different approaches like Shape Grammar \ref{sec:shape_grammar} (building faces generation), L-Systems \ref{sec:L-Systems} (plants growing), Space Syntax \ref{sec:space_syntax} and many others. 

The \textit{Chair of Information Architecture} provided a street generation and analysing tool named CPlan \ref{CPlan}. To become acquainted with the application and to allow using the existing genetic algorithms a tree generating algorithm was developed by us. 

Selecting interesting areas from different cities and recombining then into a new one would allow a fast creation of new cities. To reach this aim many steps are needed. First of all the cities must be separate into reasonable parts based on vertex position or edge length. The approach of this thesis is to use clustering algorithms from the area of machine learning. Then the different area should be valued to select useful areas.

In this document flat \ref{K-Means} (K-Means) and hierarchical \ref{hierarchicalClustering} (WPGMA, UPGMA) clustering algorithms where compared \ref{sec:measurements-speed} and extended to correct wrong assignments \ref{K-Means_shortest_path}. 

The implemented clusters are measurements \ref{sec:measurements-cluster-analysis} based on the suggestions provided by the ETH-Zurich. For further use the results can now be exported into a JSON-File. %TODO more here

Student at the ETH is working on an regrow algorithm to recombine the clusters to reasonable city.
%TODO more here

\pagebreak
\input{015_RelatedWork}

\pagebreak
\chapter{Street Network Grammars}
To analyse or generate street networks grammars can be used. In this chapter \textit{Shape Grammar}, \textit{L-Systems} and \textit{Space Syntax} are described and analysed. Because the application CPlan \ref{CPlan} did not provide the ability to use grammars and our customer preferred a graph based approach all later described algorithms are based on an unidirectional graph.

\section{Shape Grammar} \label{sec:shape_grammar}
The main key of shape grammar is to generate paintings by a new defined grammar based on shapes, selection rules, painting rules and limiting shapes. Shape grammar is a language based on an alphabet of shapes and generated shapes \citep{shapeGrammars:1972}.

A class of paintings defines the pair (S,M). S represents the shape specifications and M the material specifications. The shape specification contains a shape grammar, defining a language of two dimensional shapes, and a selection rule. M specifies a finite list of material specifications and one limiting shape on a canvas.

\subsection{Shape Grammar Definition}
\label{sec:Shape_Grammar_Definition}
A Shape Grammar is defined over an alphabet of shapes and generated n-dimensional shapes according to \citep{shapeGrammars:1972}.
\begin{quote} 
    Definition. A shape grammar (SG) is a 4-tuple: $SG = (V_T, V_M, R, I)$ where
    \begin{enumerate}
        \item $V_T$ is a finite set of shapes.
        \item $V_M$ is a finite set of shapes such that $V_T $* $\cap$  $V_M = \emptyset$
        \item R is a finite set of ordered pairs (u,v) such that

         u is a shape consisting of an element of $V_T $* combined with an element of $V_M$ and
         
         v is a shape consisting of (A) the element of $V_T $* contained in u or (B) the element of $V_T $* contained in u combined with an element of $V_M$ or (C) the element of $V_T $* contained in u with an additional element of $V_T$* and an element of $V_M$.

        \item I is a shape consisting of elements of $V_T $* and $V_M$.
    \end{enumerate}
\end{quote}

\subsection{Selection Rules}
\label{sec:Shape_Grammar_Selection_Rules}
A painting is generated based on an undefined count of shape rules. This requires a mechanism to select a correct shape. The depth is defined by levels which are being assigned during generation based on their rules: According to \citep{shapeGrammars:1972}.
\begin{displayquote}
    \begin{enumerate}
        \item The terminals in the initial shape are assigned to level 0.
        \item If a shape rule is applied, and the highest level assigned to any part ot the terminal corresponding to the level side of the rule is N, then
        \begin{enumerate}
            \item If the rule is of type A, any part of the terminal enclosed by the marker in the left side of the rule is assigned to N.
            \item If the rule is of type B, any part of the terminal enclosed by the marker in the left side of the rule is assigned to N and any part of the terminal enclosed by the marker is assigned to N+1.
            \item If the rule is of type C, the terminal added is assigned to N+1.
        \end{enumerate}
        \item No other level assignments are made.
    \end{enumerate}
\end{displayquote}

\subsection{Painting Rules}
\label{sec:Shape_Grammar_Painting_Rules}
Painting rules describe witch shape should be painted inside of a defined area. Like in a Venn diagram the rules contain multiple levels 0 - n. By combining these levels the painting colour is described\citep{shapeGrammars:1972}. The assignments are made by the common operators. An example can be viewed in figure \ref{fig:shape_grammar_gen_specifications}

\subsection{Limiting Shapes}
\label{sec:Shape_Grammar_Limiting_Shapes}
These shapes define a limiting area on the canvas, where shape painting is allowed. 
The area could have any form, but normally it is defined as a rectangle. Like a camera view the limiting shape defines the scale of a painting and its viewpoint. Therefore the initial/start shape could be outside of the limiting shape.

\subsection{Example}
\begin{figure}[!h]
    \centering
    \includegraphics{shape_grammar_generativ_spec.jpg}
    \caption{Generative Specifications - \textit{Source: Shape Grammar} \citep{shapeGrammars:1972}}
    \label{fig:shape_grammar_gen_specifications}
\end{figure}
\pagebreak
\subsubsection{Generated}
The following image \ref{fig:Shape Grammars/Example} shows the generated painting with the relevant steps. The levels are generated as described in the section \textit{selection rule} \ref{sec:Shape_Grammar_Selection_Rules} 
\newline
Level 0 is represented by steps 0 and 1. Steps 2 to 5 are on level 1 and finally steps 18 and 19 are on Level 2.
\begin{figure}[!h]
    \centering
    \includegraphics[width=\textwidth]{sg_example.jpg}
    \caption{ Generated Painting - \textit{Source: Shape Grammar} \citep{shapeGrammars:1972}}
    \label{fig:Shape Grammars/Example}
\end{figure}

\newpage
\subsection{Street Generation}
In the following section we debate the usefulness of generating street networks by shape grammar. 
\paragraph{Features}
\begin{itemize}
    \item Can describe and generate veneer in high details.
    \item With R-Shapes windows and high detail 3D-Models can be generated easily.
\end{itemize}

\paragraph{Problems}
\begin{itemize}
    \item The given methods generate monotonous streets in most cases. 
    \item A huge number of R-Shapes is required to generate a useful street network.
    \item Areas with different characteristics (historic district, rectangular raster like New York or radial to center like Paris) are difficult to generate.
    \item The R-Rules for the transitions between area characteristics should not repeat themselves and are for that reason difficult to build.
\end{itemize}

\pagebreak
\section{L-Systems} \label{sec:L-Systems}
L-System is a well established modelling approach for the synthesis of realistic plant images. There are many papers describing L-Systems and how they are applied to generate plant life: "In these cases L-System productions capture the \textit{development} of plant components over time." \citep{PrusinkiewiczEtAl:2001} Productions are applied in parallel, so that all plant parts grow and age equally. The growth is stopped at a defined terminal age. This age is the number of iterations, where in each iteration productions are applied.

The context-free productions in \citep{PrusinkiewiczEtAl:2001} are defined using the following syntax.
\begin{equation} \label{eq:lsystem context free}
    pred : \{block1\}\ cond\ \{block2\} \leadsto succ
\end{equation}
The symbol \textit{pred} (predecessor) defines the module that will get replaced by the modules defined in \textit{succ} (successor). This replacement is only applied, if the (optional) condition is met. \textit{block1} and \textit{block2} are C statement blocks, of which the first block is executed before and the second after the condition is evaluated. \citep{PrusinkiewiczEtAl:2001} gives the following example.
\begin{equation} \label{eq:lsystem example 1}
    A(x) : \{y = x + 2;\}\ y \geq 5\ \{z = y / 3;\}\ \leadsto B(z)C(z + 1)
\end{equation}
If this example production was applied to the module $A(4)$, it would result in the modules $B(2)C(3)$.

The modelling language, described in \citep{PrusinkiewiczEtAl:2001}, also supports context-sensitive productions. The following listing defines the syntax of such a production.
\begin{equation} \label{eq:lsystem context sensitive}
    lcont < pred > rcont : \{block1\}\ cond\ \{block2\} \leadsto succ
\end{equation}
\textit{lcont} (left context) and \textit{rcont} (right context) each define a list of modules that have to precede or respectively follow the \textit{pred} (module being replaced). Context modules are limited to query symbols, which are explained later on. \citep{PrusinkiewiczEtAl:2001} gives the following example.
\begin{equation} \label{eq:lsystem example 2}
    A(x) < B(y) > C(z) : x + z > 0 \leadsto M(y / 2)N(y / 2)
\end{equation}
If the example of listing \ref{eq:lsystem example 2} was applied to the module composition $A(2)B(4)C(0)$, it would result in the modules $A(2)M(2)N(2)C(0)$.

A way to generate images with L-Systems is to use a LOGO-style turtle as a graphical model. Certain modules of the L-System are interpreted as commands to this turtle.

\pagebreak
\section{Space Syntax} \label{sec:space_syntax}
The axial line-based space syntax was first introduced in 1976 by B. Hiller, A. Leaman, P. Stansall and M. Bedford in the paper \textit{Space syntax} \citep{spaceSyntax:1976}. The grammar is based on a morphic language and describes methods to analyse and generate urban areas and buildings.

The grammar was later extended and integrated into \textit{Geographic Information System} (GIS) as point-based space syntax (graph). This systems are used to plan and analyse the human interaction with the environment.

\subsection{Axial line-based space syntax and limitations}
In space syntax streets are represented by axial lines. 
\textit{"Axial lines are used to represent directions of uninterrupted movement and visibility, so they represent the longest visibility lines in two-dimensional urban spaces."} \citep{integrationSpaceSyntaxGIS:2002}
This approach has allowed many analysis methods of urban systems like way-finding process or criminal analysis. An axial map represents a fully filled free space with many axial lines \ref{fig:barnsbury_axial_map}. For a more detailed analysis the lines can by broken into segments at the intersections \ref{fig:barnsbury_segmented_axial_map}. To generate the axial map always the longest not visited space is selected and line drawn till the hole space is covered.

Axial maps have many limitations. First the complexity to create an axial map is hight because of the generating procedure. If an additional area is added the hole process of detection should be repeated because a longer start line could exist. The process is non-deterministic because a curve could be separated into a unknown count of lines.

\begin{figure}[h]
    \centering
    \begin{subfigure}[b]{0.4\textwidth}
        \includegraphics[width=\textwidth]{The-figure-ground-plan-of-Barnsbury-Axial-map.jpg}
        \caption{Barnsbury axial map}
        \label{fig:barnsbury_axial_map}
    \end{subfigure}
    \quad
    \begin{subfigure}[b]{0.4\textwidth}
        \includegraphics[width=\textwidth]{The-figure-ground-plan-of-Barnsbury-Axial-segment-map.jpg}
        \caption{Barnsbury segmented axial map}
        \label{fig:barnsbury_segmented_axial_map}
    \end{subfigure}
    \caption{Space Syntax: Axial Map - \textit{Source: UCL Space Syntax \citep{SpaceSyntaxExampels}}}
    \label{fig:SpaceSyntaxAxialMap}
\end{figure}

\subsection{Point-based space syntax}
A point-based space syntax is a decision based approach as described by Bin Jiang and Christope Claramunt in \textit{Integration of Space Syntax into GIS: New Perspectives for Urban Morphology} \citep{integrationSpaceSyntaxGIS:2002}. Every time a point in a street network is visited the visitor can decide witch direction he will travel next. The distances between two point can directly be measured and described. Every point can be assigned to an ID and marked with x and y coordinates \ref{fig:space_syntax_gis_characteristic_points}. This approach is at least equivalent to the predefined syntax created by B. Hiller et al.\citep{spaceSyntax:1976} because a visibility graph can be create to represent axial lines \ref{fig:space_syntax_gis_visibility_graph}. The point-based space syntax therefore is a graph based approach.

\begin{figure}[h]
    \centering
    \begin{subfigure}[b]{0.4\textwidth}
        \includegraphics[width=\textwidth]{space_syntax_gis_characteristic_points.jpg}
        \caption{Characteristic points}
        \label{fig:space_syntax_gis_characteristic_points}
    \end{subfigure}
    \quad
    \begin{subfigure}[b]{0.4\textwidth}
        \includegraphics[width=\textwidth]{space_syntax_gis_visibility_graph.jpg}
        \caption{Visibility graph}
        \label{fig:space_syntax_gis_visibility_graph}
    \end{subfigure}
    \caption{Point-based space syntax - \textit{Source: Integration of Space Syntax into GIS \citep{integrationSpaceSyntaxGIS:2002}}}
    \label{fig:space_syntax_gis}
\end{figure}

\subsubsection{Comparison with CPlan}
The application CPlan \ref{CPlan} does handle the street networks as graphs. As a result the same measurements methods can be applied as described for the point-based space syntax. The clustering algorithms later described \ref{sec:clustering_algorithms} are based on vertex/points and unidirectional edges between the points.

\pagebreak
\chapter{Clustering Algorithms}
\label{sec:clustering_algorithms}
In the application CPlan the street network is generated as a graph as described in \ref{CPlan}. To separate and select areas with specific characteristics the networks have been analysed by machine learning clustering algorithms. In this section the different clustering algorithms are discussed and the generated results produced by the implementation for this thesis in CPlan are compared.

\section{K-Means}  \label{sec:K-Means}
\subsection{Description}
The K-Means algorithm detects the clusters/partitions by measuring the euclidean distance between the cluster centroids and the position of the street junctions. The algorithm uses the following approach:

\begin{enumerate}
    \item A number of points are inserted into the graph and used as centroids.
    \item All graph points are assigned to their nearest centroids.
    \item The centroids are moved to the center of their assigned graph points.
    \item Until the maximum number of iterations is reached this process is repeated starting by loop item 2.
\end{enumerate}
Because the K-Means algorithm can find local minima, this process needs to be executed more than once. The result with the best found solution will be selected.

\subsection{Implementation}
The added implementation in CPlan \citep{cPlan:2015} is an optimised K-Means version named K-Means++. The idea of this algorithm is to select reasonable start centres to avoid many iterations.

\subsection{Speed Optimization}
The ETH-Zurich provided the following networks: Bad Berka (552 nodes, 626 edges), Weimar (2012 nodes, 2646 edges) and Zurich (27446 nodes, 35121 edges). The network Zurich with factor 13 more edges than Weimar resulted in long processing time. A speed improvement was  achieved by running the K-Means iterations in parallel. Every iteration can be executed free of side effects. The measurements and comparisons of the results are provided in the section Practical Task/Measurements \ref{sec:measurements};

\subsection{Result}
The implementation in CPlan \ref{CPlan} produced the following image \ref{fig:KmeansGenerated} based on the city of Weimar. All streets in one cluster are marked with the same colour, the transitions between clusters are marked black.

\begin{figure}
    \centering
    \begin{mdframed}[style=mdthight]
        \includegraphics[width=\textwidth]{clusteranalysis_kmeans_result.png}
    \end{mdframed}
    \caption{K-Means cluster analysis of Weimar
    \label{fig:KmeansGenerated}}
\end{figure}

\subsection{Problem} \label{sec:kmenasProblem}
The algorithm is based on the euclidean distance between cluster centroids and street junctions, the edge data (e.g. street length) is not used. As a result it leads to unexpected transitions between clusters. In the image \ref{fig:KmeansProblem} the result can be observed in the read circle where only one point is marked as outer cluster. The two black lines represent the cluster transitions.

\begin{figure}
    \centering
    \begin{mdframed}[style=mdthight, userdefinedwidth=0.55\textwidth, align=center]
        \includegraphics[width=\textwidth]{clusteranalysis_kmeans_problem.png}
    \end{mdframed}
    \caption{Problem of K-Means clustering
        \label{fig:KmeansProblem}}
\end{figure}

\subsection{K-Means with Shortest Path} \label{sec:K-Means_shortest_path}
To solve the problem of unexpected transitions between clusters the best result of the K-Means algorithm can be combined with the distance measurement of a \gls{APSP} algorithm (Dijkstra / Floyd-Warshall). These algorithms are used for the hierarchical clustering algorithms \ref{sec:shortest_path}.

The edges are assigned by the edge data (e.g. street length) to the nearest cluster. Therefore the result is a connected graph. This means every vertex can be reached from every other vertex within a cluster. In the generated figure \ref{fig:Kmeansshortestp} the artefacts described in \ref{sec:kmenasProblem} are removed.

Additional calculation time is needed for the shortest path algorithm. The differences can be compared in the section Measurements \ref{sec:measurements};

\begin{figure}
    \centering
    \begin{mdframed}[style=mdthight]
        \includegraphics[width=\textwidth]{clusteranalysis_kmeansExt_result.png}
    \end{mdframed}
    \caption{K-Means clustering with shortest path\label{fig:Kmeansshortestp}}
\end{figure}




\subsection{Single-Linkage Result}
The figure \ref{fig:SingleLinkage} shows the result of a hierarchical cluster analysis of Weimar using the single linkage reduction formula. The used distance function $d(i, j)$ was the shortest distance from $i$ to $j$ in the street graph. As it is clearly visible when looking at the result, this way of cluster analysis is not creating the desired output. There is one huge cluster in the middle and many one-node clusters at the border of the city.

The problem is caused by the used reduction formula. Single linkage uses the minimal distance between all nodes of the compared clusters. This leads to the creation of clusters, where all roads which connect two clusters are long. In a city, where there are multiple (direct and indirect) connections from one junction to another most of the time, this tends to create one-node clusters for nodes, which are connected to the city by a single long road.

\begin{figure}
    \centering
    \begin{mdframed}[style=mdthight]
        \includegraphics[width=\textwidth]{clusteranalysis_singlelinkage.png}
    \end{mdframed}
    \caption{Single-Linkage hierarchical cluster analysis of Weimar\label{fig:SingleLinkage}}
\end{figure}

\subsection{UPGMA and WPGMA Result}
\label{sec:UPGMAandWPGMA}
%TODO: Add figure
The figures xxx and xxx show the result of hierarchical cluster analysis using the \acrshort{UPGMA}, or respectively the \acrshort{WPGMA} reduction formula. The same distance function $d(i, j)$ as in the single linkage solution was used (shortest distance from $i$ to $j$ in the street graph).

\subsection{All Pairs Shortest Path} \label{sec:shortest_path}
As already stated, the implementations of hierarchical clustering algorithms developed during this thesis use graph distances. The calculation of the shortest path between two nodes every time it is needed would not be viable, because these distances have to be accessed multiple times. So it is necessary to calculate and store them in advance using an \gls{APSP} algorithm.
%TODO: FW, Dijkstra, FW_GPU?

\subsection{Memory usage} \label{sec:memory_usage}
The most trivial solution would be to store the distances as a matrix of double-precision floating-point numbers. This is because the given numeric data was already in this number format and a matrix is easily accessible. For \acrshort{UPGMA} and \acrshort{WPGMA} this matrix has to be extended so it also includes the cached distances between clusters.

This storage strategy would use $(2n-1)^2$ double-precision floating-point numbers, $n$ being the node count (number of junctions). As an example: For the used environment this storage strategy would amount to about 21.7 Gigabyte of used memory for the street network of Zurich.

The easiest optimisation is to convert all distances to a single-precision floating-point format. The loss of precision is not affecting the result, as mostly only the $<$ and $>$ relations of the different distances are important.

A second optimisation that can be applied is, to remove redundant information. When storing the full matrix of all distances, most values are stored twice. The matrix is mirrored along the diagonal, because the used street networks are undirected graphs. So the memory usage can almost be cut in half if those duplicates are not stored. This improvement is visualised in figure \ref{fig:memory_usage_01}.

The third and last applied optimisation was to change the way, how cluster distances are stored if the reduction formula \acrshort{UPGMA} or \acrshort{WPGMA} is used. Each time, when two clusters are combined, the distances of the new formed cluster to every other cluster are calculated. This combination step is done using the distances stored for the two old clusters. Afterwards the distances which were stored for the old clusters are no longer used. To reduce the memory usage, the distances of the new cluster can therefore be stored, at the location where the distances of one of the old clusters were stored. Figure \ref{fig:memory_usage_02} visualises this optimisation.

When applying those memory usage optimisations, $\frac{1}{2}n(n+1)$ single-precision floating point numbers have to be stored. The memory complexity still lies in $O(n^2)$, but the improvements allow most modern computers to analyse medium to big sized street networks. As an example: For the used environment the memory usage would amount to about 1.36 Gigabyte for the street network of Zurich.

\begin{figure}
    \centering
    \begin{subfigure}[b]{\textwidth}
        \includegraphics[width=\linewidth]{memoryusage01.png}
        \caption{Optimisation of memory usage by not storing redundant information. On the left hand side is the full distance matrix, on the right hand side is the same matrix without the duplicate entries.}
        \label{fig:memory_usage_01}
    \end{subfigure}
    \par\medskip
    \begin{subfigure}[b]{\textwidth}
        \includegraphics[width=\linewidth]{memoryusage02.png}
        \caption{Optimisation of memory usage by changing the cluster combination behaviour. On the left hand side the blue distances belonging to the old clusters are combined and stored in a new row (of green distances) belonging to the new formed cluster. The right hand side is similar, but the distances of the old clusters (blue and violet) are combined and stored at the location of one of the old clusters (violet). After the combination, the violet distances belong to the new formed cluster.}
        \label{fig:memory_usage_02}
    \end{subfigure}
    \caption{Visualised memory usage optimisations}
\end{figure}

\subsection{Output Modification} \label{sec:outout_modification}
In some cases the hierarchical clustering has an output of clusters with highly varying size if standard splitting order is used. (Size meaning the node / street junction count in this context.) In figure \ref{fig:unmodified_cluster_size} a clustering of Weimar with such a case is shown. In this figure there is a relatively big cluster in the centre of the city and at the bottom left corner are multiple very small clusters.

Depending on the requirements this is the desired result, as this is the optimal solution of the algorithm using the given distances. In other cases it is more important, that the created clusters are of about the same size, than it is to abide the mathematically optimal solution.

To create relatively equally sized clusters from the resulted hierarchy of a hierarchical cluster algorithm the following approach was taken. Instead of splitting the hierarchy into clusters exactly in the order in which the hierarchy was created, the order is changed. When always splitting the biggest cluster (cluster with the most nodes), a result with much more equally sized clusters will be created. A clustering with such modified sizes is shown in figure \ref{fig:modified_cluster_size}.

\begin{figure}
    \centering
    \begin{subfigure}[b]{0.49\textwidth}
        \begin{mdframed}[style=mdthight, userdefinedwidth=0.9\linewidth]
            \includegraphics[width=\linewidth]{unmodified_cluster_size.png}
        \end{mdframed}
        \caption{Clusters with original size}
        \label{fig:unmodified_cluster_size}
    \end{subfigure}
    \begin{subfigure}[b]{0.49\textwidth}
        \begin{mdframed}[style=mdthight, userdefinedwidth=0.9\linewidth]
            \includegraphics[width=\linewidth]{modified_cluster_size.png}
        \end{mdframed}
        \caption{Clusters with modified size}
        \label{fig:modified_cluster_size}
    \end{subfigure}
    \caption{Clustered street network of Weimar without \ref{fig:unmodified_cluster_size} or with \ref{fig:modified_cluster_size} cluster size modification. (Cluster count = 18, Reduction Formula = \acrshort{UPGMA})}
\end{figure}


\section{Cluster Rating}
\label{clusterRating}
To evaluate the generated clusters different measurement methods are necessary. Jun. Prof. Dr.  Reinhard König from the ETH Zurich provided the following parameters.
\newline
\begin{itemize}
    \item Geometry based measurements:
    \begin{itemize}
        \item Area based on the convex hull of the cluster area
        \item Total length of the streets
        \item Ratio between the area and total length
        \item Distribution variance of the street length
        \item Distribution variance of the street angles
        \item Ratio between street block size and surrounding circle area (minima, maxima, mean)
    \end{itemize}
    \item Centrality-based measurements normalized by the street count (minima, maxima, mean):
    \begin{itemize}
        \item In-Centrality (Integration)
        \item In-Betweenness-Centrality (Choice)
    \end{itemize}
\end{itemize}
The centrality based ratings and the block area calculation where pre implemented in CPlan \ref{CPlan}. Additional a GUI to compared and stored the results as JSON-File for each generated cluster has been added.

Some initial measurements and comparisons are made in the cluster analysis chapter \ref{sec:ClusterAnalysisMeasurements}.


\pagebreak
\chapter{CPlan}\label{CPlan}
CPlan is a tool written by the Department of Architecture of the ETH-Zurich. The goal of this application is to generate/grow street networks dynamically and extend these networks with buildings. 
\section{Improvements}
\begin{itemize}
    \item Some calls to the methods IEnumerable.ToArray() and IEnumerable.ToList() were removed. This method creates a new array / list and stores every item of the IEnumerable in this new collection. As a result the application had an extremely large footprint. To further reduce this overhead some methods were changed to take IEnumerable parameters instead of arrays.
    \item Certain graph and geometry extension methods were fixed. It would be good practice to create unit tests for such methods.
    \item The vector represented by the class Matrix2d is clockwise despite the norm is counter clockwise.
    \item Line intersections of the class Geometry2D was not correct detected an therefore corrected. 
\end{itemize}

\section{Genetic algorithms}
The ETH-Zurich already has genetic optimizations algorithms based on trees. Unfortunately they don't have a working solution to produce a tree from the existing graph. The new created tree generation produces a relative tree with absolute angles. This allowed an easier and faster work process with tree based structures.

\pagebreak
\section{Normalising Street Networks}
While testing clustering algorithms on the street network of Zurich one rough spot of this network was found: Not all streets, which lead to a junction are connected to it. As shown in figure \ref{fig:zuerich_error} floating streets exist (highlighted in purple). No end of any of these highlighted streets is connected to the rest of the street network.

To handle those floating streets a network normalisation method was developed. The normalisation snaps (unites) all junctions and street end points, which are positioned close together, into one common junction. The result of this normalisation is shown in figure \ref{fig:zuerich_fixed}.

\begin{figure}
    \centering
    \begin{subfigure}[b]{0.8\textwidth}
        \begin{mdframed}[style=mdthight]
            \includegraphics[width=\linewidth]{zuerich_street_error_cropped.png}
        \end{mdframed}
        \caption{Floating streets in the street network of Zurich}
        \label{fig:zuerich_error}
    \end{subfigure}
    \par\medskip
    \begin{subfigure}[b]{0.8\textwidth}
        \begin{mdframed}[style=mdthight]
            \includegraphics[width=\linewidth]{zuerich_street_fixed_cropped.png}
        \end{mdframed}
        \caption{Normalised street network of Zurich}
        \label{fig:zuerich_fixed}
    \end{subfigure}
    \caption{Street network of Zurich without (\ref{fig:zuerich_error}) and with (\ref{fig:zuerich_fixed}) normalisation}
\end{figure}

\section{Cluster colouring}
To visualize the clustering of a street network in this thesis, clusters are marked with different colours. This section describes the details of this cluster colouring.

The result of each clustering algorithm, which was implemented in this thesis, represents a cluster as set of vertices. Each of these vertices is a junction in the street network. The visualisation only draws streets but not the junctions, as the intersection of streets are intuitively seen as junctions. To colour the different clusters the following approach was taken:

\begin{itemize}
    \item Streets which connect two vertices that are part of the same cluster are coloured with that clusters colour.
    \item Streets which connect vertices of two distinct clusters are coloured grey.
\end{itemize}

There are papers which discuss how colours can be transformed to a perceptually uniform space, where the computation of $n$ colours with maximal distances (for the human eye) is possible \cite{colors:2006}.

In this thesis a more concise approach was taken: The papers of R. M. Boynton \cite{boynton:1989} and K. L. Kelly \cite{kelly:1965} define 11, respectively 22 colours, which are easy to distinguish by human eye. Those colours are displayed in figure \ref{fig:colours}.

Depending on the number of clusters one or the other of those colour sets (without black and white) were used. The colours were already sorted in a way that ensures the extraction of the first $n$ elements returns colours with maximal distance. If more than 20 clusters had to be coloured, the colours of Kelly were used multiple times.

\begin{figure}
    \centering
    \begin{subfigure}[b]{\textwidth}
        \begin{mdframed}[style=mdthight]
            \includegraphics[width=\linewidth]{boynton_colours.png}
        \end{mdframed}
        \caption{Boynton colours}
        \label{fig:boynton_colours}
    \end{subfigure}
    \par\medskip
    \begin{subfigure}[b]{\textwidth}
        \begin{mdframed}[style=mdthight]
            \includegraphics[width=\linewidth]{kelly_colours.png}
        \end{mdframed}
        \caption{Kelly colours}
        \label{fig:kelly_colurs}
    \end{subfigure}
    \caption{Colour palette of Boynton (\ref{fig:boynton_colours}) and Kelly (\ref{fig:kelly_colurs})}
    \label{fig:colours}
\end{figure}


\pagebreak
\chapter{Measurements}
\label{sec:measurements}
In this following chapter the generated data with CPlan \ref{CPlan} is compared and analysed. 
\section{Speed Measurements}
\label{sec:measurements-speed}
TODO: Description of speed measurements here. 

TODO: Table of speed measurements here.

\section{Cluster Analysis}
\label{sec:measurements-cluster-analysis}
In this chapter the provided measurement methods \ref{clusterRating} are used to compare different districts/areas. The following images were generated with the cluster algorithm FastUPGMA \ref{sec:UPGMAandWPGMA} on Weimar with \textit{Modified Output} and \textit{Number of Clusters} count 16. 

\subsection{Historic District}
\label{sec:historyDistinct}
This district is characterised by short streets with many connections. As a result the block areas are small and the block count per area is high. Additionally mean integration and mean choice values are high. This can be observed in the image \ref{fig:historic_district} and the measured data in table \ref{sec:ClusterAnalysisMeasurements} in column C1.
%TODO BILD

\begin{figure}
    \centering
    \begin{subfigure}[b]{0.6\textwidth}
        \begin{mdframed}[style=mdthight]
            \includegraphics[width=\linewidth]{historic_district.png}
        \end{mdframed}
        \caption{Historic District of Weimar}
        \label{fig:historic_district}
    \end{subfigure}
    \par\medskip
    \begin{subfigure}[b]{0.6\textwidth}
        \begin{mdframed}[style=mdthight]
            \includegraphics[width=\linewidth]{business_district.png}
        \end{mdframed}
        \caption{Business District of Weimar}
        \label{fig:business_district}
    \end{subfigure}
    \begin{subfigure}[b]{0.6\textwidth}
        \begin{mdframed}[style=mdthight]
            \includegraphics[width=\linewidth]{outskirts_district.png}
        \end{mdframed}
        \caption{Outskirts Area of Weimar}
        \label{fig:outskirts_district}
    \end{subfigure}
    \caption{Different areas of Weimar. Historic District (\ref{fig:historic_district}), Business District (\ref{fig:business_district}) and Outskirts Area (\ref{fig:outskirts_district})}
\end{figure}

\subsection{Business/Manhattan District}
\label{sec:businessDistinct}
If the block count and the connection count is high the given area is probably a business/Manhattan district. The street have a high mean connection count and the block area to circle area is high. The image \ref{fig:business_district} with the measured data in table \ref{sec:ClusterAnalysisMeasurements} in column C2, is an example area of this district/area type.

\subsection{Outskirts Area}
\label{sec:outskits}
These areas are characterized by extreme long streets and a low connection count as you can see in the example image \ref{fig:outskirts_district} and the mesured data in table \ref{sec:ClusterAnalysisMeasurements} in column C3. The block count is compared with a business or historic district exceptional low.


\subsection{Measured Data}
\label{sec:ClusterAnalysisMeasurements}
The following table \ref{tab:cluterAnalysisDescription} contains the parameters with additional descriptions. Extended information can be found below the table. Of every parameter the minimal (min), maximal (max), mean (average) and the median value can be calculated.

\begin{table}[h]
\begin{center}
    \begin{tabular}{ | l | l |} \hline 
        Parameter & Description \\ 
        \hline
        Total Area &  Area of the convex hull \\ \hline
        Total Length & Sum of the street length \\ \hline
        Density & Total Area divided to Total Length  \\ \hline
        
        Street Length Min/Max/Mean & Shortest/Longest/Average street length  \\ \hline
        Street Length Median & Middle value of the length dataset \\ \hline
        Street Length Variance & Sigma of the normal distribution curve of the variance \\ \hline
        
        Vertex Connections & Mean connected edges per vertex  \\ \hline
        
        Street Angle Min/Max/Mean & Smallest/Biggest/Average angle between two edges \\ \hline
        Street Angle Variance & Sigma of the normal distribution curve of the angles \\ \hline
        
        Block Count & Total number of blocks \\ \hline
        Block Area Min/Max/Mean & Shortest/Biggest/Average block area \\ \hline
        Block Area A/Ac Min/Max/Mean & Block area divided to a minimal circle around a block \\ \hline
        
        Integration Min/Max/Mean & Normalised In-Centrality \\ \hline
        Choice Min/Max/Mean & Normalised In-Betweenness-Centrality \\ \hline
    \end{tabular}
    \caption{Parameter with descriptions for table \ref{tab:measured_cluster_ratings}}
    \label{tab:cluterAnalysisDescription}
\end{center}
\end{table}


\subsubsection{Block Area}
In the paper \textit{A typology of street patterns}\citep{blockArea:2014} the method is described how cities/areas can be classified and compared by analysing the block areas instead of the streets. First the block area is calculated and then the result is divided by the circumscribed circle area.

\subsubsection{Centrality}
The parameter \textit{Integration} describes the closeness centrality. This means the normed sum of the distances from all other vertex based on a shortest path algorithm is calculated. The vertex with the lowest value must be the most central.

With \textit{Choice} the betweenness centrality is described. The approach is to calculate the shortest path between every vertex. Every time a given vertex is visited the betweenness-centrality value of the vertex is raised by one. As a result the highest measured value indicates for a vertex to be in or near the centre of a graph.

\subsubsection{Variance}
The parameter variance describes the sigma of the normal curve of the distribution function.
First of all the measured data is round to fit into a distribution function \ref{eq:distribution_function}. Then the expected value \ref{eq:expected_value} and the variance \ref{eq:variance} is calculated. Finally the standard deviation (square root of V(x)) is computed \ref{eq:standard_deviation}.
\begin{align}
\label{eq:distribution_function} 
F(x) &= P(X \leq x) =  \sum_{t\in{X}, t\leq{x}}{f(t)} \\
\label{eq:expected_value} 
E(x) &= \int\limits_{-\infty}^\infty x * f(x)dx \\
\label{eq:variance} 
V(x) &= \int\limits_{-\infty}^\infty (x - E(X))^2 * f(x)dx \\
\label{eq:standard_deviation} 
\sigma(x) &= \sqrt{V(x)}
\end{align}

\begin{table}[h]
\begin{center}
\begin{tabular}{ |l|l|l|l|l|l| }
    \hline
    Parmater &  
    & C1 \ref{sec:historyDistinct} 
    & C2 \ref{sec:businessDistinct} 
    & C3 \ref{sec:outskits}
    & C4  \\ 
    \hline
    \multirow{4}{*}{Total} 
    & Area & 1838.05 & 1956.59 & & \\
    & Length & 806.92 & 643.50 & & \\
    & Density & 2.28 & 3.04 & & \\ 
    \hline
    \multirow{5}{*}{Street Length}
    & Min & 0.66 & 0.69 & & \\
    & Max & 9.13 & 10.68 & & \\
    & Mean & 2.72 & 3.85 &  & \\ 
    & Median & 2.30 & 3.65 & & \\ 
    & Variance & 1.70 & 2.19 & & \\ 
    \hline
    \multirow{1}{*}{Vertex} 
    & Connections & 3.04 & 2.84 & & \\
    \hline
    \multirow{5}{*}{Street Angle} 
    & Min & 0.00 & 0.00 & & \\
    & Max & 151.80 & 358.84 & & \\
    & Mean & 119.03 & 123.75 & & \\ 
    & Variance & 124.68 & 131.65 & & \\ 
    \hline
    \multirow{5}{*}{Block} 
    & Count & 55 & 35 & & \\
    & Area Min & 0.01 & 0.09 & & \\
    & Area Max & 31.75 & 42.29 & & \\
    & Area Mean & 6.54 & 13.65 & & \\ 
    & A/Ac Min & 0.01 & 0.01 & & \\
    & A/Ac Max & 0.54 & 0.55 & & \\
    & A/Ac Mean & 0.18 & 0.23 & & \\ 
    \hline
    \multirow{5}{*}{Integration} 
    & Min & 0.46 & 0.48 & & \\
    & Max & 0.57 & 0.75 & & \\
    & Mean & 0.49 & 0.59 & & \\ 
    \hline
    \multirow{5}{*}{Choice}
    & Min & 0.00 & 0.00 & & \\
    & Max & 1.00 & 0.67 & & \\
    & Mean & 0.10 & 0.05 & & \\ 
    \hline
\end{tabular}
\caption{Measured results from Cluster1 \ref{sec:historyDistinct}, Cluster2 \ref{sec:businessDistinct}, Cluster3 \ref{sec:outskits} and Cluster4}
\label{tab:measured_cluster_ratings}
\end{center}
\end{table}


\pagebreak
\chapter{Future Work}
\label{sec:future_work}
More detailed clustering
Detect cluster with other algorithms.
Insert additional measurement methods.

CPlan
Change all libs to 64 versions

ETH: 
Student remove one cluster and regrow from edges.
Recombine different clusters.
Use the exported mesurements to select the correct districts/areas.

TODO Not completed!

\pagebreak
\chapter{Conclusion}
Reached good results with clustering methods 
Best results with ...

Cluster analysis produces comparable results.

Corrected errors in application

\bibliography{quotations}
\appendix
\printnoidxglossaries
\end{document}