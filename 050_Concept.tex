\chapter{Concept}
\section{Tree Creation}
To create a tree from an existing graph the angle and distance should be enough data to fully represent a graph.
Two approaches exist to define angles. One is to separate the circle range into minus and plus 180 degrees. This would allow to step straight forward with the angle 0 degree (Figure \ref{fig:tree-generation-decision} left). Another approach is to use the range from 0 to 360 degrees. In this case straight forward would be 180 degrees (Figure \ref{fig:tree-generation-decision} right).

\begin{figure}[!ht]
    \centering
    \includegraphics[width=\textwidth]{tree-generation-decision.png}
    \caption{Degree separation method \label{fig:tree-generation-decision}}
\end{figure}

Because a tree can not fully represent a graph, many points exist multiple times in the resulting tree. To recreate a graph, junctions should be added where streets intersect without junctions. And the resulting graph should be snapped where points are at the same position. Additionally the existing multiple points in the graph should be removed.

\FloatBarrier
\pagebreak
\section{K-Means}
Points are assigned to different clusters by using the minimal distances. The following approach is used:

\begin{enumerate}
    \item First, random centroids are created.
    \item Then every point is assigned to the nearest centroid by calculating the euclidean distance.
    \item Then centroids are moved to the center of their assigned graph points.
    \item Till no point is moved or the maximum iterations is reached the process is repeated from item 2.
\end{enumerate}

\subsection{Connected Cluster Problem} \label{sec:kmenasProblem}
The K-Mean algorithm is based on the euclidean distance between cluster centroids and street junctions, the edge data (e.g. street length) is not used. As a result it leads to unexpected transitions between clusters.

\subsection{Connected Cluster Approach} \label{sec:connected_cluster_approach}
To solve the problem of unexpected transitions between clusters the best result of the K-Means algorithm can be combined with the distance measurement of a \acrlong{APSP} (\acrshort{APSP}) algorithm (Dijkstra / Floyd-Warshall). These algorithms are used for the hierarchical clustering algorithms \ref{sec:shortest_path}.

\subsection{Additional Features}
%TODO Explain
TODO

\pagebreak
\section{Hierarchical Clustering}
%TODO Short version what singe linkage & PGMA does

\subsection{Hight Memory Usage}
Because WPGMA / UPGMA needs to store all data the memory footprint would be very high. %TODO Describe why exacter.

\subsection{Output Modification}
Idea of output modification
%TODO Describe Idea of output modification!

\pagebreak
\section{Cluster Analysis}
To measure and compare the different areas they should be characterized. Therefore some districts with noticeable characteristics haven been selected an compared to each other. The found characteristics could help to separate a given city on a feature based approach.

%TODO compare corretly only between images not measurements! 

\subsection{Historic District}
\label{sec:historyDistinct}
This district is characterised by a hight count of short streets with many connections. As a result the block areas are small and the block count per area is high. Additionally the mean angles are hight and the density (convex hull area divided by total street length) is low. This characteristics can be observed in the image \ref{fig:historic_district}.

\begin{figure}[!ht]
    \centering
    \begin{mdframed}[style=mdthight, userdefinedwidth=0.4\textwidth, align=center]
        \includegraphics[width=\linewidth]{district_historic.png}
    \end{mdframed}
    \caption{Historic District of Weimar}
    \label{fig:historic_district}
\end{figure}

\FloatBarrier
\subsection{Business/Manhattan District} 
\label{sec:businessDistinct}
If the relative block area (block area divided by surrounding circle area) is high the given subgraph is probably a business/Manhattan district. The mean connection count and the density compared with a historic district \ref{sec:historyDistinct} should be much higher. An example business district can be found on the map of Weimar \ref{fig:business_district}.

\begin{figure}[!ht]
    \centering
    \begin{mdframed}[style=mdthight, userdefinedwidth=0.4\textwidth, align=center]
        \includegraphics[width=\linewidth]{district_business.png}
    \end{mdframed}
    \caption{Business District of Weimar}
    \label{fig:business_district}
\end{figure}

\FloatBarrier
\subsection{Outskirts Area}
\label{sec:outskits}
These areas are characterized by extreme long streets and a low connection count. As a result the density is very high. The block count is compared with a business or historic district exceptional low. In the image \ref{fig:outskirts_district} these characteristics can be found.

\begin{figure}[!ht]
    \centering
    \begin{mdframed}[style=mdthight, userdefinedwidth=0.4\textwidth, align=center]
        \includegraphics[width=\linewidth]{district_outskirts.png}
    \end{mdframed}
    \caption{Outskirts Area of Weimar}
    \label{fig:outskirts_district}
\end{figure}