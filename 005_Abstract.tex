\begin{abstract}
    In this document different grammars to describe buildings, streets or plants were discussed, analysed and evaluated for their usefulness.

    To separate street networks into reasonable parts different clustering algorithms were implemented and tested. First the centroid-based (K-Means) algorithm produced reasonable regions but not every cluster was a connected subgraph. For more precise results hierarchical clustering algorithms (Single-Linkage, WPGMA and UPGMA) were realised and the results compared. To reduce the memory footprint specialised data structures were used.

    An analysing method to select useful areas like the city centre or a business area were then created. It allows to select different clusters and compare measurements like the block area mean or the density (total area divided by total street length).
\end{abstract}
