\begin{abstract}
    In this paper different cluster analysis algorithms where compared and extended on a graph representation of a street networks.
    
    The centroid based K-Means algorithm produced reasonable clusters unfortunately with wrong assignments. Because the assignments where vertex based not every cluster was connected. To correct this error a shortest path algorithm was used to get connected clusters.
    
    For hierarchical algorithms there are many different reduction formula algorithms. 
    First the Single Linkage algorithm produced one huge cluster in the middle and many at the border. This algorithm used the distances of the nearest clusters and therefore in a city with multiple connections one big cluster in the middle was created.
    
    Then the reduction algorithms UPGMA/WPGMA used also the distances between the smaller clusters. The result was many clusters with different sizes. To produce cluster with equal sizes the output was modified to merge with the smaller cluster.
    
    The hight memory usage lead to problems. To reduce the memory footprint float precision was used. Further a data structure witch does not store value twice was used. Then the resulting cluster distances where saved at the position of the resource positions. 
    
    Analysis of create clusters
    
\end{abstract}
